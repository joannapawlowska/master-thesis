\documentclass[12pt]{article}

\usepackage{polski}
\usepackage{amsmath}
\usepackage{float}

% breake table between multiple pages
\usepackage{longtable}

% always add indent
\usepackage{indentfirst}

% turn on hyperlinks to the bibliography when \cite{}
\usepackage{hyperref}

% set margins
\usepackage{geometry}
\newgeometry{tmargin=2.5cm, bmargin=2.5cm, lmargin=3.5cm, rmargin=1.5cm}

% set spacing (to set '1.5' insert '1.3')
\linespread{1.3}

% add a dot after any section number
\usepackage{titlesec}
\titlelabel{\thetitle.\quad}


\begin{document}

% set size of font in section and subsection
% \Large -> 17.28 pt (the closest to 16 pt)
% \large -> 14.4 pt (the closest to 14 pt)
\titleformat*{\section}{\Large\bfseries}
\titleformat*{\subsection}{\large\bfseries}

\section*{Wstęp}

\section{Rozdział pierwszy}

\section{Rozdział drugi}

\newpage

\section{Przegląd literatury}

Rozdział trzeci zawiera przegląd publikacji naukowych. Tematem omawianych prac są różne metody programowania równoległego stosowane
w algorytmach uczenia maszynowego.

\subsection{Równoległe konstrukcje algorytmów drzew decyzyjnych}

W artykule \cite{parallel-implementation-decision-tree} zostało przedstawionych kilka strategii konstrukcji algorytmów drzew decyzyjnych, które oparte są o techniki takie jak: równoległość zadań,
równoległość danych oraz równoległość hybrydowa. W pracy zaprezentowana została autorska implementacja równoległej konstrukcji drzewa
decyzyjnego algorytmem C4.5. Na zakończenie autorzy przedstawili wyniki działania algorytmu i postawili wstępne wnioski dotyczące jego działania.

Artykuł rozpoczyna się od zaprezentowania trudności, które pokazują, jak złożonym zadaniem jest implementacja równoległych
algorytmów do budowy drzew decyzyjnych. Wymienione zostają m.in. problemy z zastosowaniem statycznego, jak i dynamicznego przydzielania procesorów.
Nieregularny kształt drzewa, który określany jest dopiero w momencie działania programu, jest dużą przeszkodą do stosowania statycznej alokacji. Takie podejście prowadzi
najczęściej do nierównomiernego rozłożenia obciążenia. W przeciwnej sytuacji, gdy dane przetwarzane są przez dynamicznie przydzielane procesory, problemem staje się
konieczność zaimplementowania przekazywania danych. Współdzielenie danych jest wymagane, ponieważ część danych związanych z~rodzicami musi dostępna być również dla potomków.

Autorzy szczegółowo opisują różnice pomiędzy równoległością zadań, równoległością danych oraz równoległością hybrydową. Równoległość zadań określana jest
jako dynamiczne rozdzielanie węzłów decyzyjnych między procesory, w celu kontynuowania ich rozbudowy. Wadą takiego podejścia jest konieczność replikacji całego zbioru
treningowego lub, alternatywnie, zapewnienie dużej ilości komunikacji pomiędzy procesorami. Równoległość danych przedstawiona jest jako wykonywanie tego samego zbioru instrukcji
algorytmu przez wszystkie zaangażowane procesory. Zbiór treningowy zostaje podzielony (pionowo lub poziomo) pomiędzy procesory tak, że każdy z nich odpowiedzialny jest za
inny zestaw przykładów ze zbioru. Autorzy zwracają uwagę, że przetwarzanie z pionowym podziałem danych narażone jest na~wystąpienie nierównowagi obciążenia.
Równoległość hybrydowa scharakteryzowana jest jako połączenie równoległości zadań oraz danych. Dla węzłów, które muszą przetworzyć dużą liczbę przykładów, wykorzystywana
jest równoległość danych. W ten sposób unika się problemów związanych z nierównomiernym obciążeniem. W przypadku węzłów z przypisaną mniejszą ilością przykładów
czas, potrzebny do komunikacji może być większy niż czas potrzebny do przetwarzania przykładów. Zastosowanie równoległości zadań w takie sytuacji pozwala uniknąć dysproporcji.

W kolejnej części artykułu przestawiona została implementacja równoległej konstrukcji drzewa decyzyjnego. Program został stworzony do wykonywania w środowisku pamięci
rozproszonej, w której każdy z procesorów ma~własną pamięć prywatną. Autorzy zaproponowali takie podejście, ponieważ ma ono rozwiązywać dwa problemy wspominane na
początku pracy: równoważenie obciążenia oraz konieczność przekazywania danych. Każdy z~procesorów ma za zadanie tworzyć własne listy atrybutów i klas na podstawie
przydzielonych podzbiorów przykładów. Wykorzystanie obydwu list jest kluczem do osiągnięcia efektywnego paralelizmu. Wpisy w liście klas zawierają etykietę klasy, indeks
globalny przykładu w zbiorze treningowym oraz~wskaźnik do węzła w drzewie, do którego należy dany przykład. Listy atrybutów również zawierają wpis dla każdego przykładu
z atrybutem, jak również indeks wskazujący na odpowiadający wpis w liście klas. Każdy procesor znajduje własne, najlepsze podziały lokalnego zbioru dla każdego atrybutu.
Następnie komunikuje się z pozostałymi procesorami, w celu ustalenia jednego, najlepszego podziału. Po podziale (utworzeniu węzła) następuje aktualizacja list atrybutów
przez każdy procesor, dokonana poprzez rozdzielenie atrybutów w zależności od wartości wybranego atrybutu dzielącego.

Zaprezentowane przez autorów wyniki określone są jako wstępne i wymagające udoskonaleń. Autorzy zdecydowali się jednak na wykorzystanie ich~do~przewidzenia oczekiwanego
zachowania algorytmu. Implementacja wykorzystuje takie same kryteria oceny jak stosowane w algorytmie C4.5, dlatego autorzy skupili się głównie na analizie czasu potrzebnego do
zbudowania drzewa. Do wszystkich testów wykorzystany był zestaw danych syntetycznych Agrawal, w którym każdy przykład ma dziewięć atrybutów (pięć ciągłych i~trzy dyskretne).

Z przedstawionych rezultatów testów wynika, że algorytm wykazał dobre wyniki przyspieszania. Twórcy artykułu przeprowadzili również testy mające na celu sprawdzenie skalowalności.
Jak w pierwszym przypadku, testy wykazały, że algorytm osiąga dobre wyniki skalowania.

\subsection{Równoległe uczenie zespołowe sieci neuronowych i lokalnych wzorców binarnych}

W pracy \cite{parallel-ensemble-learning} został zaprezentowany sposób pozwalający na rozpoznawanie twarzy.
Podejście oparte jest o równoległe uczenie zespołowe lokalnych wzorców binarnych (LBP) oraz konwolucyjnych sieci neuronowych (CNN). Metoda LBP zastosowana została
do ekstrakcji cech tekstury twarzy, które posłużyły jako dane treningowe sieci CNN.

Paralelizm w omawianym przykładzie został uzyskany poprzez zastosowanie równoległego uczenia zespołowego. Kilka konwolucyjnych sieci neuronowych, opartych o różne
struktury LBP, zostało wykorzystanych do uczenia się, a następnie klasyfikowania zestawów danych treningowych. Każda sieć kończyła trening podając wynik klasyfikacji. Ostateczny
wynik uzyskiwany był na podstawie głosowania większościowego. Wyjściowy wynik sieci o największej liczbie głosów przyjmowany był jako finalny klasyfikator obrazu twarzy.

W celu sprawdzenia skuteczności przedstawionego rozwiązania autorzy zdecydowali się na wybór dwóch zestawów danych: Yale-B i ORL. Zbiór Yale-B składa się 576 obrazów 38 twarzy
o różnych wyrazach, które wykonane zostały w zmiennych warunkach oświetlenia. Zbiór ORL składa się z 40 obrazów 4 twarzy, po 10 zdjęć na osobę. Współczynnik rozpoznawania
zdefiniowany został jako stosunek liczby skutecznie rozpoznanych obrazów do całkowitej ilości obrazów w zbiorze testowym. W pracy zintegrowanych zostało 10 sieci neuronowych.
Sprzęt, na którym przeprowadzono testy, posiadał następujące parametry: CPU Intel(R) Core(TM) i5-.6300HQ 2.3GHZ, pamięć RAM 8G DDR4, karta graficzna NVIDIA GeForce GTX 960M,
system operacyjny Windows 10 64 bity oraz środowisko programistyczne Python 3.5 Tensorflow-gpu 1.10.0.

Dokładność zintegrowanych sieci dla zbioru treningowego ORL wyniosła około 98\%. Po przeprowadzeniu 2800 treningów dokładność zbioru testowego zwiększyła się aż 100\%, co daje
znacznie lepszy wynik niż w przypadku zastosowania pojedynczej sieci konwolucyjnej. Dowodzi to wysokiej odporności zintegrowanych sieci na zmiany postawy ciała czy oświetlenia.
Dla zbioru Yale-B dokładność pojedynczej sieci zbliżona była do 85\%, natomiast zintegrowanej sieci wyniosła około 97,5\%. Na podstawie wyników przeprowadzonych testów
autorzy stwierdzili, że dokładność sieci neuronowej jest znacznie zmniejszona, gdy do klasyfikacji nie~wykorzystuje się schematu uczenia zespołowego.

\subsection{Równoległe implementacje algorytmu genetycznego}

Autorzy artykułu \cite{parallelized-genetic-algorithms} zaprezentowali trzy różne implementacje równoległych algorytmów gentycznych(GA):
Master-Slave, Coarse-Grained oraz Fine-Grained. Modele zostały zaimplementowane przy użyciu języka programowania Python, brokera wiadomości RabbitMQ
oraz modułu ,,Scalable Concurrent Operations in Python" (SCOOP) umożliwiającego programowanie równoległe.

W pierwszej części artykułu autorzy przedstawiają różnice pomiędzy równoległymi implementacjami a tradycyjną implementacją sekwencyjną. Model Master-Slave w wersji synchronicznej
działa prawie tak jak model sekwencyjny. Różni się tylko w przetwarzaniu funkcji Fitness, które rozdzielone jest pomiędzy różne procesory. Model Master-Slave może
zwiększyć szybkość GA poprzez równomierne rozłożenie obciążenia między procesorami, mimo konieczności komunikacji pomiędzy nimi. W modelu Fine-Grained tworzona jest jedna
globalna populacja rozproszona przestrzennie na węzły (procesory). W ten sposób zostaje utworzona topologia z sąsiedztwami. Sąsiedztwa tworzą przestrzeń, w której odbywa się równoległa i
tylko lokalna selekcja (osobnik uczestniczy w selekcji tylko w obrębie sąsiedztwa). Z powodu izolacji sąsiedztwa, najlepsze osobniki rozprzestrzeniają się wolnej niż w innych rodzajach GA,
co zwiększa różnorodność populacji. Dodatkowo tylko jeden element centralny w obrębie jednego sąsiedztwa jest poddawany modyfikacjom przez krzyżowanie i mutacje. Do zalet tego modelu
autorzy zaliczają dużą wydajność. Model Coarse-Grained różni się od modelu Fine-Grained głównie tym, że pracuje on z mniej rozdrobnioną globalną populacją (w przypadku Fine-Grained
jest to po jeden lub dwa osobniki na węzeł). Autorzy przyjmują zasadę, że model Coarse-Grained występuje wtedy, gdy liczba węzłów jest mniejsza niż liczba osobników w jednym z nich.

Paralelizacja GA została otrzymana poprzez użycie funkcji z modułu SCOOP, natomiast do komunikacji między procesami wybrany został serwer RabbitMQ. Testy pomiędzy poszczególnymi
implementacjami a implementacją szeregową zostały przeprowadzone na następującym sprzęcie: Linux z~systemem operacyjnym Fedora wersja 25, CPU Intel(R) Xeon(R) L5408
2.13 GHz z czterema procesorami, pamięć RAM 32 GB z wirtualizacją trzech innych stacji roboczych z systemem operacyjnym Ubuntu 17.10. Rozmiar populacji rozpoczynał się od
64 osobników, a następnie zwiększany był dla kolejnych testów.

Zgodnie z oczekiwaniami autorów, najmniejsze zapotrzebowanie na pamięć operacyjną wykazał model sekwencyjny. Model Coarse-Grained charakteryzował się natomiast największym
zużyciem pamięci. Pomiędzy modelami równoległymi niższym zużyciem CPU wyróżnił się model Master-Slave. Efektywność samego algorytmu oceniana jest w artykule poprzez porównanie
liczby iteracji potrzebnych do znalezienia najlepszego rozwiązania. We~wszystkich równoległych modelach GA można było zaobserwować zależność pomiędzy liczbą iteracji, wielkością
populacji oraz zużyciem pamięci. Wraz ze~wzrostem ilości osobników liczba iteracji malała, natomiast wzrastało zużycie pamięci. Najszybszym modelem okazał się model Fine-Grained,
który uzyskiwał najlepszy wynik prawie 27-krotnie szybciej niż model sekwencyjny. Testy przeprowadzone przez autorów potwierdziły korzyści płynące z wykorzystania równoległych
implementacji algorytmu genetycznego. Wszystkie trzy modele równoległe osiągnęły istotne przyspieszenie i lepszą wydajność w porównaniu z modelem sekwencyjnym. W szczególności
modele Fine-Grained i Coarse-Grained były bardziej wydajne, ponieważ liczba wymaganych iteracji była znacznie mniejsza niż w modelu sekwencyjnym.

\subsection{Zestawienie artykułów}

W literaturze można znaleźć wiele artykułów na temat metod programowania równoległego wykorzystywanego do optymalizacji pracy algorytmów uczenia maszynowego. 
Metody jednak mogą znacznie się od siebie różnić w zależności od rodzaju algorytmu, dla którego są przeznaczone. W dalszej części pracy uwaga zostanie poświęcona
już tylko podejściom stosowanym w algorytmach do budowy drzew decyzyjnych.

Tabela \ref{table:articles-parallel-decision-tree} zawiera odnośniki do tytułów artykułów, słowa kluczowe oraz krótki opis przybliżający tematykę poruszaną w każdym z artykułów.

    % \renewcommand{\arraystretch}{1.5}
    % \setlength{\tabcolsep}{0.07\textwidth}
    \begin{center}

        \begin{longtable}{| c | p{0.19\textwidth} | p{0.62\textwidth} |}

            \hline
            
            \textbf{Artykuł} &\textbf{Słowa kluczowe} & \multicolumn{1}{|c|}{\textbf{Opis}}
            
            \\ \hline \hline 

            \cite{parallelization-of-decision-tree-al} &

            równoległość wewnątrzwęzłowa; równoległość międzywęzłowa &

            Autorzy przedstawili oraz porównali wydajność czterech metod do równoległej implementacji algorytmu C4.5.
            W~analizie uwzględniony został rodzaj danych wykorzystywanych do konstrukcji drzewa, który okazał
            się mieć duży wpływ na wyniki wydajności porównywanych metod. Wyszczególnione zostały trzy rodzaje
            techniki wewnątrzwęzłowe, które wykorzystują zrównoleglanie przetwarzania danych. Równolegle
            przetwarzane mogą być rekordy, atrybuty lub ich kombinacja (podejście hybrydowe). W technice międzywęzłowej
            równolegle przetwarzane są całe węzły -- wszystkie operacje, które muszą zostać przeprowadzone do stworzenia
            węzła, wykonywane są przez jeden wątek. \\
            
            \hline

            \cite{parallel-hoeffding-decision-tree} &

            węzły Hoeffdinga; architektura Master-Slave; &

            Autorzy artykułu zdecydowali się na zrównoleglanie tylko fragmentów algorytmu tj. szukania
            najlepszego atrybuty do podziału węzła. Zadania przydzielane przez główny procesor
            (master) innym procesorom (slave) polegają na obliczenia zysku informacyjnego dla określonego
            atrybutu. Zastosowana architektura oraz zrównoleglanie tylko wyszukiwania atrybutów skutkuje
            jednak w niskiej skalowalności. Pomimo tego testy wykazały, że przyspieszenie działania algorytmu
            jest dość efektywne. Dzięki zastoso- \\
            
            & 

            &
            waniu nierówności Hoeffdinga, do podziału drzewa nie muszą być
            używane wszystkie dane. Ma to dodatkowy wpływ na szybkość działania algorytmu. \\
            
            \hline

            \cite{parallel-algorithm-to-induce-decision-trees} &

            równoległość w węźle; duże zbiory danych &

            W artykule przedstawiono algorytm ParDTLT. Algorytm jest równoległą wersją algorytmu
            DTLT (Decision Trees from Large Training sets), który nie wymaga ładowania w całości wszystkich danych
            treningowych do pamięci komputera. ParDTLT oparty jest na idei sekwencyjnej budowy struktury
            drzewa oraz równoległym przetwarzaniu danych w każdym węźle. W danym czasie wszystkie procesory
            dostępne są dla jednego węzła. W węźle uprzywilejowanym tworzona jest kolejka atrybutów, dla
            których obliczany jest zrównoważony zysk informacji. Analiza kolejnych
            atrybutów przydzielane są procesorom do momentu, aż kolejka będzie pusta. Pozostałe węzły drzewa czekają, aż
            staną się węzłem uprzywilejowanym. Autorzy artykuły przeprowadzili testy algorytmu, które wykazały, że
            ParDTLT jest szybszy od algorytmów jak np. DTLT czy C4.5. \\

            \hline

            \cite{improved-id3-decision-tree} &

            wielowątkowość; pamięć \newline współdzielona; algorytm ID3 &

            Modyfikacjom został poddany algorytm ID3. Przedstawiono dwie różne implementacje wykorzystujące równoległość.
            Pierwsza polega na stworzeniu tylu wątków, ile istnieje atrybutów, dla których obliczony musi zostać
            przyrost informacji. Gdy atrybut dzielący zostanie odnaleziony, tworzony jest węzeł. Następnie ponownie
            tworzone są kolejne wątki, które przetwarzają atrybuty nowo powstałych węzłów. Dopiero gdy wszystkie wątki
            zakończą pracę, z dostarczonych wyników składane jest drzewo. Różnica w drugiej implementacji polega na 
            wykorzystaniu pamięci współdzielonej, dzięki czemu algorytm jest bardziej wydajnym pamięciowo. 
            Węzeł główny jest współdzielony, dlatego każdy nowy węzeł może być tworzony, gdy tylko atrybut dzielący
            został odnaleziony. \\
            
            \hline

        \caption{Zestawienie artykułów poruszających tematykę równoległości w algorytmach drzew decyzyjnych}
        \label{table:articles-parallel-decision-tree}
 
        \end{longtable}

    \end{center}

\begin{thebibliography}{6}

    \bibitem{parallel-implementation-decision-tree} Amado, N., Gama, J., \& Silva, F. (2001). Parallel Implementation of Decision Tree Learning Algorithms. Lecture Notes in Computer Science, 6–13. doi:10.1007/3-540-45329-6\_4 
    \bibitem{parallel-ensemble-learning} Tang, J., Su, Q., Su, B., Fong, S., Cao, W., \& Gong, X. (2020). Parallel Ensemble Learning of Convolutional Neural Networks and Local Binary Patterns for Face Recognition. Computer Methods and Programs in Biomedicine, 105622. doi:10.1016/j.cmpb.2020.105622
    \bibitem{parallelized-genetic-algorithms} Skorpil, V., Oujezsky, V., \& Tuleja, M. (2020). Testing of Python Models of Parallelized Genetic Algorithms. 2020 43rd International Conference on Telecommunications and Signal Processing (TSP). doi:10.1109/tsp49548.2020.9163475
    \bibitem{parallelization-of-decision-tree-al} Kubota, K., Nakase, A., Sakai, H., \& Oyanagi, S. (2000). Parallelization of decision tree algorithm and its performance evaluation. Proceedings Fourth International Conference/Exhibition on High Performance Computing in the Asia-Pacific Region. doi:10.1109/hpc.2000.843500
    \bibitem{parallel-hoeffding-decision-tree} Cal, P., \& Woźniak, M. (2013). Parallel Hoeffding Decision Tree for Streaming Data. Advances in Intelligent Systems and Computing, 27–35. doi:10.1007/978-3-319-00551-5\_4
    \bibitem{parallel-algorithm-to-induce-decision-trees} Rcega, A. F.-A., Suarez-Cansino, J., \& Flores-Flores, L. G. (2013). A parallel algorithm to induce decision trees for large datasets. 2013 XXIV International Conference on Information, Communication and Automation Technologies (ICAT). doi:10.1109/icat.2013.6684045 
    \bibitem{improved-id3-decision-tree} Maheshwari, S., Jatav, VK., \& Meena, YK. (2011). Improved ID3 Decision Tree Generation using Shared-Memory and Multi-Threading Approach. 2011 International Conference on Education Technology and Computer (ICETC 2011). doi:10.13140/2.1.3216.2247

\end{thebibliography}


\end{document}